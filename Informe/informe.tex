\documentclass[a4paper,12pt]{report}
\usepackage[utf8]{inputenc}
\usepackage{graphicx}
\usepackage{amsmath}
\usepackage{hyperref}
\usepackage{bookmark}

%
\begin{document}
\title{Sistema de Recomendación Híbrido}
\author{Carlos A. Bresó Sotto \\
Marian S. Álvarez Suri}

\urldef{\repo}\url{https://github.com/marians002/IRS-Final-Project.git}

\date{Facultad de Matemática y Computación \\ 
           Universidad de La Habana, La Habana 
           \\ \repo}

\maketitle

\begin{abstract}
    En el ámbito de los sistemas de recomendación, los enfoques híbridos han demostrado ser altamente efectivos al combinar las fortalezas de múltiples técnicas de recomendación. Este trabajo presenta un sistema de recomendación híbrido que integra métodos basados en contenido y filtrado colaborativo para mejorar la precisión y relevancia de las recomendaciones. El sistema propuesto utiliza un modelo de filtrado colaborativo para capturar las preferencias de los usuarios a partir de sus interacciones pasadas, mientras que el enfoque basado en contenido analiza las características intrínsecas de los ítems para ofrecer recomendaciones personalizadas. La combinación de estos métodos se realiza mediante un esquema de ponderación adaptativa que ajusta la contribución de cada técnica según el contexto y el perfil del usuario. Los resultados experimentales muestran que el sistema híbrido supera a los enfoques individuales en términos de precisión y satisfacción del usuario, destacando su potencial para aplicaciones en diversas áreas como el comercio electrónico, la recomendación de películas y la educación en línea.
\\
\\
    \textbf{Keywords}: Sistema de Recomendación Híbrido, filtrado colaborativo, filtrado basado en contenido.
\end{abstract}

\tableofcontents

\chapter{Introducción}
    \section{Motivación}

    Los sistemas de recomendación han emergido como uno de los pilares fundamentales en la estrategia digital de las grandes compañías a nivel mundial, transformándose en herramientas indispensables para mejorar la experiencia del usuario y maximizar la eficiencia de los servicios en línea. Empresas líderes en el sector tecnológico, como Amazon, Netflix y Spotify, han adoptado estos sistemas para ofrecer a sus usuarios recomendaciones personalizadas, lo que no solo incrementa la satisfacción y fidelidad del cliente, sino que también impulsa significativamente las ventas y el compromiso. La capacidad de estos sistemas para analizar grandes volúmenes de datos en tiempo real y predecir las preferencias de los usuarios con alta precisión les permite a las compañías estar un paso adelante, asegurando una oferta más ajustada a las necesidades y gustos de su audiencia. En un mundo cada vez más saturado de información y opciones, los sistemas de recomendación se han convertido en una herramienta clave para filtrar y personalizar el contenido, demostrando ser un recurso valioso para mantener la competitividad y liderazgo de las grandes empresas en el mercado global.

    El proyecto en cuestión aborda la creación de un sistema de recomendación híbrido, una solución innovadora que combina las fortalezas del filtrado colaborativo y el filtrado basado en contenido para superar las limitaciones inherentes a cada uno de estos enfoques por separado. El filtrado colaborativo aprovecha las interacciones y preferencias de los usuarios para hacer recomendaciones, basándose en la premisa de que aquellos con gustos similares en el pasado tendrán intereses parecidos en el futuro. Por otro lado, el filtrado basado en contenido se centra en las características de los ítems para hacer recomendaciones, lo que permite una personalización más detallada, pero puede limitar la diversidad de las recomendaciones y reforzar la cámara de eco.
    Al integrar ambos enfoques, nuestra propuesta no solo mejora la precisión de las recomendaciones, sino que también enriquece la experiencia del usuario al ofrecer una variedad más amplia y relevante de opciones.

    \section{Objetivos}

    Nuestro trabajo se distingue de otros en el campo de los sistemas de recomendación por la incorporación de un enfoque innovador que mejora la diversidad de los elementos recomendados. A diferencia de los métodos tradicionales que se centran únicamente en las preferencias explícitas del usuario, hemos añadido un conjunto de recomendaciones que no se encuentran entre las preferidas por este. Dicha estrategia no solo enriquece la experiencia del usuario al introducirle nuevos y variados contenidos, sino que también mitiga el problema de la homogeneidad en las recomendaciones, fomentando así una exploración más amplia de las opciones disponibles. 

    El objetivo general de este trabajo es desarrollar un sistema de recomendación híbrido para películas que no solo se base en las preferencias explícitas del usuario, sino que también incorpore elementos que podrían resultar atractivos a pesar de su diferencia con el historial previo o con los gustos de personas con intereses similares. Esta estrategia busca enriquecer la diversidad de las recomendaciones cinematográficas, permitiendo descubrir nuevas películas y opciones que de otro modo podrían pasar desapercibidas.


    \section{Estructura del Informe}

\chapter{Estado del Arte}
    \section{Sistemas de Recomendación}
    En el ámbito de los sistemas de recomendación híbridos, el estado del arte ha evolucionado significativamente, reflejando un creciente interés en superar las limitaciones de los enfoques tradicionales mediante la integración de múltiples métodos de recomendación. Según el análisis presentado en \cite{hybridRecommenderSystemReview}, los sistemas de recomendación híbridos ofrecen una solución efectiva para superar las limitaciones de los sistemas basados únicamente en filtrado colaborativo o contenido, pues han demostrado ser especialmente eficaces en abordar problemas como el arranque frío y la escasez de datos, ofreciendo recomendaciones más precisas y personalizadas. Los estudios destacan la diversidad de técnicas utilizadas para combinar diferentes enfoques, desde métodos ponderados hasta la fusión de características, mostrando una tendencia hacia la experimentación y la innovación en el campo \cite{recommendationModelsTourism} . 
    Existen varias formas de combinar estos métodos, entre las cuales destacan:
    \begin{enumerate}
        \item Ponderación de Pesos 
        \item Mezcla de Modelos
        \item Sistema Híbrido Alternante
        \item Combinación de Características
        
    \end{enumerate}


    Estas revisiones sistemáticas subrayan la importancia de los sistemas de recomendación híbridos en el panorama actual, identificando áreas clave para la investigación futura, como la mejora de los algoritmos de fusión y la exploración de nuevas fuentes de datos para enriquecer las recomendaciones. La continua evolución de estos sistemas promete no solo mejorar la experiencia del usuario en plataformas digitales, sino también abrir nuevas vías para el avance tecnológico en la inteligencia artificial y el análisis de datos \cite{systematicReviewRecommenderSystems}.

    \subsection{Filtrado Colaborativo}
    \subsection{Filtrado Basado en Contenido}
    \subsection{Sistemas de Recomendación Híbridos}
    \section{Métodos de Hibridación}
        \subsection{Ponderación de Pesos}
        En este enfoque, las recomendaciones de diferentes sistemas se combinan asignando un peso a cada una de ellas. La importancia o puntuación de cada producto se calcula a partir de los resultados obtenidos por todas las técnicas de recomendación presentes en el sistema. Un ejemplo de este tipo de hibridación es el sistema P-Tango \cite{recommendationModelsTourism}, que ajusta los pesos de acuerdo con las opiniones de los usuarios sobre los resultados mostrados.
        \subsection{Mezcla de Modelos}
        En este método, las recomendaciones provenientes de varias técnicas de recomendación se muestran al mismo tiempo. Esto permite que el usuario vea una combinación de recomendaciones basadas en diferentes enfoques, lo que puede mejorar la diversidad y la calidad de las recomendaciones.
        \subsection{Sistema Híbrido Alternante}
        Este enfoque utiliza criterios específicos para alternar entre diferentes técnicas de recomendación. Por ejemplo, primero se emplean técnicas basadas en contenido y, si no se obtienen buenos resultados, el sistema cambia a filtrado colaborativo. Esto permite que las recomendaciones sean sensibles a las fortalezas y debilidades de cada método \cite{knowledgeDiscoveryMethodology}.
        \subsection{Combinación de Características}
En este método, las características de las fuentes de datos de varias técnicas de recomendación se combinan en un único algoritmo de recomendación. Esto permite aprovechar la información de diferentes fuentes para generar recomendaciones más precisas.
\chapter{Metodología}
\section{Descripción del Sistema}
El sistema desarrollado es un recomendador híbrido que combina múltiples modelos de recomendación con pesos específicos. Este sistema se encarga de generar recomendaciones personalizadas para los usuarios basándose en sus interacciones previas con los ítems, en la similitud con otros usuarios, y en la diversidad de géneros. 

Además, se ha implementado una funcionalidad que mejora las sugerencias finales incorporando opciones derivadas de películas con géneros distintos a los inicialmente seleccionados, enriqueciendo así las propuestas presentadas al usuario.\section{Datos Utilizados}
Se utiliza el conjunto de datos MovieLens 100K, que contiene 100,000 calificaciones de películas realizadas por 943 usuarios sobre 1682 películas. Este conjunto de datos incluye información sobre las películas, como títulos, géneros y fechas de lanzamiento, así como las calificaciones otorgadas por los usuarios \cite{movielens}.

\section{Implementación}
\subsection{Algoritmos Utilizados}
Se tomaron los siguientes algoritmos de recomendación de la librería cornac \cite{cornac} para implementar el sistema híbrido:
\begin{itemize}
    \item SVD (Singular Value Decomposition): Un modelo de factorización de matrices que descompone la matriz de calificaciones en matrices de características latentes.
    \item ItemKNN (Item-based K-Nearest Neighbors): Un modelo basado en la similitud entre ítems, utilizando la similitud coseno.
    \item BPR (Bayesian Personalized Ranking): Un modelo de optimización para el ranking personalizado basado en la probabilidad bayesiana.
\end{itemize}

\subsection{Integración de Técnicas}
El sistema se considera híbrido porque integra algoritmos de dos paradigmas principales de recomendación: el filtrado colaborativo y el basado en contenido. Esta combinación permite aprovechar las fortalezas y mitigar las debilidades de cada enfoque individualmente, ofreciendo recomendaciones más precisas y relevantes.

Para combinar los modelos mencionados utilizamos un esquema de ponderación. Los pesos asignados a cada modelo son (4, 1, 6) para SVD, BPR y ItemKNN, respectivamente. Estos pesos fueron elegidos porque recompensan positivamente a los algoritmos que mejor desempeño tuvieron individualmente al recomendar ítems novedosos.

Además, se calcula la similitud entre las películas basándose en sus géneros utilizando la similitud de coseno. Esta información de similitud se utiliza para mejorar las recomendaciones, añadiendo películas no tan similares a las mejor recomendadas.


\chapter{Resultados y Discusión}
    \section{Evaluación del Sistema}
        \subsection{Métricas de Evaluación}

        Para evaluar el rendimiento del sistema de recomendación híbrido, se utilizan varias métricas estándar en el campo de los sistemas de recomendación \cite{metrics}. Estas métricas proporcionan una evaluación cuantitativa de la precisión, relevancia y diversidad de las recomendaciones generadas por el sistema, permitiendo una comparación objetiva con otros enfoques y configuraciones.
        Las métricas empleadas se explican a continuación:

        \begin{enumerate}
            \item MAE (Mean Absolute Error): MAE mide el error promedio absoluto entre las calificaciones predichas y las calificaciones reales. Un MAE bajo indica que las predicciones del sistema están cercanas a las valoraciones reales. Se elige por su interpretabilidad directa en términos de error promedio en las predicciones.
            \item RMSE (Root Mean Square Error): RMSE es la raíz cuadrada del error cuadrático medio. Penaliza más los errores grandes que el MAE, lo que significa que un RMSE alto indica la presencia de errores grandes en algunas predicciones. Se selecciona por su sensibilidad a los errores grandes, lo que es crucial en sistemas de recomendación donde es importante minimizar las malas recomendaciones.
            \item Precision@10: Precision@10 mide la proporción de recomendaciones relevantes entre las top-10 recomendaciones hechas por el sistema. Una Precision@10 alta indica que la mayoría de las top-10 recomendaciones son relevantes para el usuario. Se elige para evaluar la calidad y relevancia de las recomendaciones en la parte superior de la lista, lo cual es importante desde la perspectiva del usuario, ya que los usuarios suelen prestar atención solo a las primeras recomendaciones.
            \item Recall@10: Recall@10 mide cuántos ítems relevantes se encuentran entre las top-10 recomendaciones del sistema, respecto al total de ítems relevantes. Un Recall@10 alto indica que el sistema es capaz de identificar una gran proporción de los ítems relevantes en sus top-10 recomendaciones. Se utiliza para asegurar que el sistema no solo haga recomendaciones precisas sino que también capture una buena parte de los ítems relevantes en sus recomendaciones principales.
            \item Novedad (para evaluar la diversidad): Aunque no es una métrica estándar como las anteriores, la novedad puede medir cuán desconocidos o nuevos son los ítems recomendados para el usuario. Se puede adaptar para evaluar la diversidad al considerar cómo de variadas son las recomendaciones entre sí. La inclusión de la novedad para evaluar la diversidad asegura que el sistema no solo recomiende ítems relevantes sino que también promueva el descubrimiento y la exploración al sugerir ítems variados y potencialmente inesperados, mejorando la experiencia del usuario.
        \end{enumerate}


Estas métricas fueron elegidas para proporcionar una evaluación comprensiva del sistema de recomendación, abarcando la precisión de las predicciones (MAE, RMSE), la calidad de las recomendaciones (Precision@10, Recall@10), y la capacidad del sistema para ofrecer experiencias de descubrimiento a través de recomendaciones diversas y novedosas.




        \subsection{Resultados Obtenidos}

        En esta sección, presentamos una comparativa exhaustiva de las métricas de evaluación obtenidas por los modelos empleados. La Tabla \ref{tab:metrics} resume los resultados en términos de Error Medio Absoluto (MAE), Raíz del Error Cuadrático Medio (RMSE), Precisión y Recall considerando los primeros 10 ítems recomendados (Precision@10 y Recall@10), así como el tiempo de entrenamiento y de prueba (en segundos) para cada modelo. Estos resultados nos permiten analizar la eficacia y eficiencia de cada enfoque en el contexto de nuestro sistema de recomendación híbrido comparándolos con métodos tradicionales como Factorización de Matrices (MF), BPR (Bayesian Personalized Ranking), SVD (Descomposición de Valores Singulares) y ItemKNN (K-Nearest Neighbors para ítems).
    \begin{table}[h]
        \centering
        \caption{Evaluación combinada del sistema de recomendación}
        \label{tab:metrics}
        \begin{tabular}{lrrrrrrr}
        \hline
        \textbf{Modelo} & \textbf{MAE} & \textbf{RMSE} & \textbf{Precision@10} & \textbf{Recall@10} & \textbf{Novedad} \\ \hline
        MF      & 0.7435 & 0.9023 & 0.0672 & 0.0464  & 0.02052 \\
        BPR     & 2.1572 & 2.3603 & 0.1138 & 0.1120 & 0.00254 \\
        SVD     & 0.7494 & 0.9084 & 0.0574 & 0.0397 & 0.02707 \\
        ItemKNN & 0.8222 & 0.9944 & 0.0305 & 0.0176 & 0.26961 \\
        Hybrid  & 0.7968 & 0.9476 & 0.1356 & 0.1042 & 0.32069 \\ \hline
        \end{tabular}
    \end{table}
    \section{Análisis de Resultados}

    \subsection{Precisión}
    \begin{itemize}
    \item MAE y RMSE: El sistema MF (Matrix Factorization) muestra los valores más bajos de MAE y RMSE, indicando que, en términos de error de predicción, es el más preciso. El sistema híbrido tiene un MAE y RMSE mayores que MF y SVD pero menores que BPR e ItemKNN, lo que sugiere que, aunque no es el más preciso en términos de error de predicción, supera a varios otros métodos.
    
    \item Precision@10: El sistema híbrido lidera con una Precision@10 de 0.1356, lo que indica que tiene la mayor proporción de recomendaciones relevantes entre las top-10 recomendaciones. Por tanto, desde la perspectiva de la precisión de las recomendaciones, el sistema híbrido es superior.
    \end{itemize}
    

    \subsection{Diversidad}
    En el análisis de la métrica de novedad, empleada para medir la diversidad en nuestro sistema de recomendación, observamos resultados significativos que reflejan la capacidad de cada modelo para introducir ítems novedosos a los usuarios. La puntuación media de novedad para el modelo MF fue de 0.02052, mientras que para el modelo BPR fue notablemente baja, con un valor de 0.00254, indicando una menor tendencia a recomendar ítems menos conocidos. Por otro lado, el modelo SVD mostró una mejora en la novedad con una puntuación de 0.02707. Sin embargo, es en los modelos ItemKNN y Hybrid donde vemos un salto significativo en la capacidad de ofrecer recomendaciones novedosas, con puntuaciones de 0.26961 y 0.32069, respectivamente. Estos resultados sugieren que, mientras los modelos MF, BPR y SVD pueden tener un enfoque más conservador, centrado en la precisión de las recomendaciones, los modelos ItemKNN y especialmente el sistema híbrido se destacan por su habilidad para explorar y recomendar una gama más amplia y diversa de ítems.

    \subsection{Comparación con Otros Sistemas}

    \begin{itemize}
        \item Contra MF y SVD: Aunque el sistema híbrido no supera a MF y SVD en términos de MAE y RMSE, sí ofrece mejores resultados en Precision@10 y en la métrica de novedad. Esto sugiere que, aunque MF y SVD pueden ser más precisos en predecir las calificaciones, el sistema híbrido es más efectivo en identificar ítems relevantes para incluir en las top-10 recomendaciones y en promover una mayor diversidad en las recomendaciones a través de ítems novedosos.
        \item Contra BPR e ItemKNN: El sistema híbrido supera claramente a BPR y ItemKNN en todas las métricas de precisión consideradas, así como en la métrica de novedad. Esto indica que el sistema híbrido no solo es más preciso en sus predicciones y en la relevancia de sus recomendaciones, sino que también sobresale en ofrecer recomendaciones más diversas y novedosas, enriqueciendo la experiencia del usuario con una variedad más amplia de ítems.
    \end{itemize}
        
\chapter{Conclusiones y Trabajo Futuro}
    \section{Conclusiones}

    El sistema híbrido, aunque no es el más preciso en términos de predicción de calificaciones (MAE y RMSE), demuestra ser el más efectivo en ofrecer recomendaciones relevantes (Precision@10). Además, su buen desempeño en Recall@10 sugiere que es capaz de capturar una amplia gama de ítems relevantes, lo que podría indicar una mayor diversidad en sus recomendaciones. Esto lo hace particularmente valioso en escenarios donde la relevancia y la diversidad de las recomendaciones son críticas para la experiencia del usuario.
    \section{Limitaciones}
    \section{Trabajo Futuro}




\bibliographystyle{IEEEtran}
\bibliography{references}

\end{document}
